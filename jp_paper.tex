\documentclass[a4paper,11pt,twocolumn]{article}
\usepackage[top=1.2in,bottom=1.2in,left=1.2in,right=1in]{geometry}

\title{From Floyd to SPFA}
\date{\vspace{-5ex}}
\author{Jiahao Li}

\begin{document}

\bibliographystyle{plain}

\twocolumn[
\begin{@twocolumnfalse}
\maketitle

\begin{abstract}
Tokenization is the process of dividing Chinese sentence romanization into tokens that represent the romanizations of each Chinese character. This paper proposes a statistical algorithm based on dynamic programming to address the tokenization problem in the design of Chinese, and more specifically, Cantonese input methods. It is an algorithm with $\Theta(n^3)$ time complexity, in which $n$ is length of input romanization sequence. In the end of the paper, the accuracy of the algorithm is evaluated based on several Chinese language corpora.
\end{abstract}
\end{@twocolumnfalse}
]

\section{introduction}


\bibliography{jp_paper.bib}

\end{document}
